\chapter{Introduction}
\label{cha:intro}

The COVID-19 pandemic is having a huge impact on our lives, that goes beyond the direct effects of the virus. Besides the fear of infection, lockdown measures adopted by many countries are limiting the possibility to move, work, have contact with others, and are creating a situation of economic crisis and generalized uncertainty about the future. The psychological effects of this unprecedented situation need to be studied.

\paragraph{Context and motivations}

During this year, everyone's daily life changed significantly and we had to adapt to restrictive measures in order to stop the disease, whether we liked it or not. The research proposed by Eurecat (Centro Tecnológico de Catalunya) really caught my eye: the possibility to study how people perceived all of this situation and to better understand which measures were more welcomed than others was really fascinating and, above all, may be useful in the case of some other unfortunate event.

\section{Project description}
\label{sec:project}

The project consisted in an analysis of emotions as emerging from Twitter messages during the pandemic.

Lexicon-based sentiment analysis tools have been employed to characterize emotions associated with content on a large scale. Moreover, users have been divided based on their gender, to study the different emotional response of males and females, their age, to understand the impact of the pandemic on different age brackets, and also their location, to analyze users' emotions considering a particular place.

This could allow us to contrast the emotional reaction with the evolution of contagions and deaths, and with the different lockdown and de-escalation stages, in different areas.

\section{Twitter}
\label{sec:twitter}

Twitter is an American microblogging and social networking service created by Jack Dorsey, Noah Glass, Biz Stone, and Evan Williams in March 2006 and launched in July of that year~\cite{enwiki:1027840990}.

Registered users can perform operations similar to those available on other social networks, e.g. create (tweet), like, or share (retweet) a post; however, these are public, and even unregistered users can read them. 

In particular, users post and interact with particular messages known as “tweets”, which have a limited number of characters. Tweets were originally restricted to 140 characters, but the limit was doubled to 280 for non-CJK languages.

As of Q1 2019, Twitter had more than 330 million monthly active users, and is considered a some-to-many microblogging service because the vast majority of tweets are written by a small minority of users.

\paragraph{Why did we use Twitter data?}

The main reason behind this critical choice is the fact that retrieving data from Twitter is particularly easy. That is because, in the majority of the case, posts are public and everyone can see them (i.e. there are less privacy related issues). Furthermore, it is possible to have access to a considerable amount of data, which is fundamental for this kind of research. 

Obviously, the data from other platforms (e.g. Facebook, Instagram, Reddit, other minor blogs, \ldots) could have been interesting. However, it is either too difficult to get the data (due to particular limitations) or to get enough data. For this reason, we believed Twitter was the best possible option.

Furthermore, given its relevant in social and political debate, Twitter can be considered like a kind of online “public sphere”~\cite{doi:10.1080/1369118X.2012.756050}.

On the other hand, Twitter maximum number of characters per tweet limits the possibilities of the users to express their feelings: this could have a negative impact on the performance of sentiment analysis. However, with a sufficient amount of data, is possible to reduce this to a bare minimum. 

\section{Sentiment analysis}
\label{sec:sentiment-analysis}

Sentiment analysis (also known as opinion mining or emotion AI) is the use of natural language processing, text analysis, computational linguistics, and biometrics to systematically identify, extract, quantify, and study affective states and subjective information~\cite{enwiki:1024880646}. Sentiment analysis is widely applied to voice of the customer materials such as reviews and survey responses, online and social media, and healthcare materials for applications that range from marketing to customer service to clinical medicine.

The objective and challenges of sentiment analysis can be shown through some simple examples:
\begin{itemize}
	\item I do not dislike carrots. (Negation handling)
	\item There are times when I regret not being a cat (Adverbial modifies the sentiment)
	\item It's all day that I was waiting to clean my room! (Possibly sarcastic)
	\item I think that the best part of the movie is when the villain dies. (Negative term used in a positive sense in certain domains).
	\item \ldots
\end{itemize}

A basic task in sentiment analysis is classifying the polarity of a given text at the document, sentence, or feature/aspect level - whether the expressed opinion in a document, a sentence or an entity feature/aspect is positive, negative, or neutral. Advanced, “beyond polarity” sentiment classification looks, for instance, at emotional states such as enjoyment, anger, disgust, sadness, fear, and surprise.

\paragraph{Emotion detection}

Also called emotion recognition, is the process of identifying human emotions~\cite{enwiki:1023798177}. 

To solve this particular task, different approaches have been developed. In particular, I had the possibility to use knowledge-based techniques (sometimes referred to as lexicon-based techniques), where domain knowledge and the semantic and syntactic characteristics of the language are used to detect emotions. In practice, it is possible to use dictionaries to map  terms to the emotions. 

One of the advantages of this approach is the accessibility of such knowledge-based resources. On the other hand, is limited and, for example, cannot properly handle complex linguistic rules.

\section{Related work}
\label{sec:related-work}

\begin{itemize}
	\item Gabriele Etta, Matteo Cinelli, Alessandro Galeazzi, Carlo Michele Valensise, Mauro Conti and Walter Quattrociocchi. News consumption and social media regulations policy, 2021
	\item Mattia Mattei, Guido Caldarelli, Tiziano Squartini and Fabio Saracco. Italian Twitter semantic network during the Covid-19 epidemic, 2021
	\item Hannah Metzler, Bernard Rimé, Max Pellert, Thomas Niederkrotenthaler, Anna Di Natale, and David Garcia. Collective Emotions during the COVID-19 Outbreak, 2021
\end{itemize}